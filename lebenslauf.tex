%%%%%%%%%%%%%%%%%
% This is an sample CV template created using altacv.cls
% (v1.7.2, 28 August 2024) written by LianTze Lim (liantze@gmail.com). Compiles with pdfLaTeX, XeLaTeX and LuaLaTeX.
%
%% It may be distributed and/or modified under the
%% conditions of the LaTeX Project Public License, either version 1.3
%% of this license or (at your option) any later version.
%% The latest version of this license is in
%%    http://www.latex-project.org/lppl.txt
%% and version 1.3 or later is part of all distributions of LaTeX
%% version 2003/12/01 or later.
%%%%%%%%%%%%%%%%

%% Use the "normalphoto" option if you want a normal photo instead of cropped to a circle
% \documentclass[10pt,a4paper,normalphoto]{altacv}

\documentclass[10pt,a4paper,ragged2e,withhyper]{altacv}
%% AltaCV uses the fontawesome5 and simpleicons packages.
%% See http://texdoc.net/pkg/fontawesome5 and http://texdoc.net/pkg/simpleicons for full list of symbols.

% Change the page layout if you need to
\geometry{left=1.25cm,right=1.25cm,top=1.5cm,bottom=1.5cm,columnsep=1.2cm}

% The paracol package lets you typeset columns of text in parallel
\usepackage{paracol}
% \usepackage[backend=biber]{biblatex}

% Change the font if you want to, depending on whether
% you're using pdflatex or xelatex/lualatex
% WHEN COMPILING WITH XELATEX PLEASE USE
% xelatex -shell-escape -output-driver="xdvipdfmx -z 0" sample.tex
\iftutex
  % If using xelatex or lualatex:
  \setmainfont{Roboto Slab}
  \setsansfont{Lato}
  \renewcommand{\familydefault}{\sfdefault}
\else
  % If using pdflatex:
  \usepackage[rm]{roboto}
  \usepackage[defaultsans]{lato}
  % \usepackage{sourcesanspro}
  \renewcommand{\familydefault}{\sfdefault}
\fi

% Change the colours if you want to
\definecolor{SlateGrey}{HTML}{2E2E2E}
\definecolor{LightGrey}{HTML}{666666}
\definecolor{DarkPastelRed}{HTML}{450808}
\definecolor{PastelRed}{HTML}{8F0D0D}
\definecolor{GoldenEarth}{HTML}{E7D192}
\colorlet{name}{black}
\colorlet{tagline}{PastelRed}
\colorlet{heading}{DarkPastelRed}
\colorlet{headingrule}{GoldenEarth}
\colorlet{subheading}{PastelRed}
\colorlet{accent}{PastelRed}
\colorlet{emphasis}{SlateGrey}
\colorlet{body}{LightGrey}

% Change some fonts, if necessary
\renewcommand{\namefont}{\Huge\rmfamily\bfseries}
\renewcommand{\personalinfofont}{\footnotesize}
\renewcommand{\cvsectionfont}{\Large\sffamily\bfseries}
\renewcommand{\cvsubsectionfont}{\large\bfseries}


% Change the bullets for itemize and rating marker
% for \cvskill if you want to
\renewcommand{\cvItemMarker}{{\small\textbullet}}
\renewcommand{\cvRatingMarker}{\small\faCircle}
% ...and the markers for the date/location for \cvevent
% \renewcommand{\cvDateMarker}{\faCalendar*[regular]}
% \renewcommand{\cvLocationMarker}{\faMapMarker*}


% If your CV/résumé is in a language other than English,
% then you probably want to change these so that when you
% copy-paste from the PDF or run pdftotext, the location
% and date marker icons for \cvevent will paste as correct
% translations. For example Spanish:
% \renewcommand{\locationname}{Ubicación}
% \renewcommand{\datename}{Fecha}


%% Use (and optionally edit if necessary) this .tex if you
%% want to use an author-year reference style like APA(6)
%% for your publication list
% \input{pubs-authoryear.cfg}

%% Use (and optionally edit if necessary) this .tex if you
%% want an originally numerical reference style like IEEE
%% for your publication list

%% sample.bib contains your publications

\usepackage{tikz}
\usepackage[german]{babel}
\usepackage{csquotes}
\usepackage[
  backend=biber,
  style=numeric,        % or `authoryear` if you prefer
  sorting=ydnt          % year descending, name, title
]{biblatex}

\addbibresource{publications.bib}

\definecolor{LeftMarginColor}{HTML}{F28C26}

% Draw the colored margin on every page
\AddToHook{shipout/background}{%
  \begin{tikzpicture}[remember picture, overlay]
    \fill[LeftMarginColor]
      (current page.south west) rectangle
      ([xshift=1.00cm]current page.north west);
  \end{tikzpicture}
}

\definecolor{LeftColumnColor}{HTML}{F7F5FA}

\AddToHook{shipout/background}{%
  \begin{tikzpicture}[remember picture, overlay]
    % Fill the left column background
    \fill[LeftColumnColor]
      ([xshift=1.00cm]current page.north west) rectangle
      ([xshift=6.10cm]current page.south west);  % 1.25cm + 3cm = 8.25cm
  \end{tikzpicture}
}

\renewcommand{\divider}{\textcolor{PastelRed!50}{\hdashrule{1.015\linewidth}{0.6pt}{0.6ex}}\medskip}

\ExplSyntaxOn
\csdef{email symbol}{\large\faEnvelope}
\ExplSyntaxOff

\ExplSyntaxOn
\csdef{linkedin symbol}{\large\faLinkedin}
\ExplSyntaxOff

\ExplSyntaxOn
\NewDocumentCommand{\multilineaddress}{m}{
  \noindent
  {\large\textcolor{PastelRed}{\faHome}}%
  \seq_set_split:Nnn \l_tmpa_seq { ; } { #1 }
  \int_set:Nn \l_tmpa_int { \seq_count:N \l_tmpa_seq }
  \seq_map_indexed_inline:Nn \l_tmpa_seq {
    \int_compare:nTF { ##1 == 1 }
      {\hspace*{0.0em}##2}
      {\hspace*{1.77em}##2}
    \int_compare:nF { ##1 == \l_tmpa_int } { \\[0.5em] }
  }
  \vspace{1em}
}
\ExplSyntaxOff

\NewInfoField{gender}{\large\faMars}

\renewcommand{\cvsection}[2][]{%
  \nointerlineskip\bigskip% bugfix in v1.6.2
  \ifstrequal{#1}{}{}{\marginpar{\vspace*{\dimexpr1pt-\baselineskip}\raggedright\input{#1}}}%
  {\color{heading}\cvsectionfont{#2}}\\[-1ex]%
  \makebox[0pt][l]{\color{headingrule}\rule{\dimexpr\linewidth+0.25cm}{1pt}}\par\medskip
}

\begin{document}
% \name{Your Name Here}
% \tagline{Your Position or Tagline Here}
%% You can add multiple photos on the left or right
%\photoR{2.8cm}{Globe_High}
% \photoL{2.5cm}{Yacht_High,Suitcase_High}

% \makecvheader
%% Depending on your tastes, you may want to make fonts of itemize environments slightly smaller
% \AtBeginEnvironment{itemize}{\small}

%% Set the left/right column width ratio to 6:4.
\columnratio{0.25}

% Start a 2-column paracol. Both the left and right columns will automatically
% break across pages if things get too long.
\begin{paracol}{2}

\cvsection{Persönliche Daten}

\email{sks.sauravs@yahoo.com}\\[1em]
\linkedin{sksdotsauravs}\\[1em]
\multilineaddress{WH 762-11-01-17; Siegmunds Hof 2-4; 10555 Berlin; Deutschland}\\
\gender{Männlich}

\vspace{1.5em}

\cvsection{Kenntnisse und\\[0.3em] Fähigkeiten}

Python\\[1em]
R\\[1em]
Java\\[1em]
HTML\\[1em]
CSS\\[1em]
NumPy\\[1em]
Pandas\\[1em]
PyTorch\\[1em]
HuggingFace\\[1em]
Flair\\[1em]
Matplotlib\\[1em]
Plotly\\[1em]
Spring Framework\\[1em]
Apache Maven\\[1em]
MySQL\\[1em]
Redis\\[1em]
Hibernate\\[1em]
JUnit\\[1em]
Mockito\\[1em]
Git\\[1em]
Jenkins\\[1em]
Rundeck\\[1em]
Docker\\[1em]
Kubernetes\\[1em]
Grafana\\[1em]

\vspace{1.5em}

\cvsection{Sprachen}

\cvskill{{\normalfont \textcolor{body}{Deutsch}}}{2.5}
\vspace{1em}

\cvskill{{\normalfont \textcolor{body}{Englisch}}}{4.5}
\vspace{1em}

\cvskill{{\normalfont \textcolor{body}{Bengalisch}}}{5}


\switchcolumn

\begin{flushleft}
  {\Huge\bfseries\textcolor{PastelRed}{Saurav Kumar Saha}}\\[1pt]
\end{flushleft}
\vspace{1em}


\cvsection{Berufserfahrung}

\cvevent{Studentische Hilfskraft - NLP und Datenschutzforschung}{Cognitive Algorithms Lab, BHT \href{https://calgo-lab.de/}{\faGlobe}}{Dezember 2024 -- März 2025}{Berlin, Deutschland}
\begin{itemize}
\item \justifying Entwurf und Implementierung eines \textbf{einheitlichen Pseudonymisierungstools} für deutsche Texte unter Verwendung des \textbf{mT5}-Sprachmodells
\item \justifying Bewertung der Modellleistung durch \textbf{Benchmarking} mit klassischen \textbf{NER}-Modellen und \textbf{LLM Prompt-Engineering}-Techniken für 10K+ deutsche E-Mail-Texte
\item \justifying Einsatz einer leichtgewichtigen Textpseudonymisierungs-App mit \textbf{REST API} (FastAPI) und \textbf{Webinterface} (Streamlit) auf einem universitären \textbf{Kubernetes}-Cluster, die die Generierung von 5 pseudonymisierten Varianten pro Benutzereingabe ermöglicht
\item \justifying Mitautor eines \textbf{Forschungspapiers}, das auf der ersten internationalen Konferenz über KI in Medizin und Gesundheitswesen in Medicine and Healthcare (AiMH 2025) angenommen wurde, das sich mit dem \textbf{Datenschutz} im klinischen NLP beschäftigt, \faGithub\ \href{https://github.com/calgo-lab/pseugc}{GitHub}
\end{itemize}

\vspace{0.75em}

\divider

\vspace{0.75em}

\cvevent{Werkstudent - Softwareentwicklung}{Solactive AG \href{https://www.solactive.com/}{\faGlobe}}{Februar 2020 -- August 2023}{Berlin, Deutschland}
\begin{itemize}
\item \justifying Entwicklung einer automatisierten Pipeline für die Erstellung von Financial \textbf{Factsheets} mit Python, Docker und Rundeck - Reduktion der Lieferzeiten um 90\% von Stunden auf Minuten
\item \justifying Einführung einer lokalen Entwicklungsumgebung mit \textbf{Docker}, um die Einarbeitungszeit für neue Entwickler um 70\% zu reduzieren und manuelle Einrichtungsschritte zu vermeiden
\item \justifying Entwicklung von mehr als 5 \textbf{Jenkins}-Testautomatisierungspipelines, wodurch der manuelle Testaufwand in anderen Teams um etwa 60\% reduziert wurde, was sich direkt auf die Software-Release-Zyklen auswirkte
\item \justifying Implementierung einer Funktion zum Zurücksetzen der Zeit für die Plattform Test Systems, einer auf \textbf{Microservices} basierenden Java-Anwendung, die mit \textbf{Spring Boot} und \textbf{MySQL} erstellt wurde und das Zurücksetzen von mehr als 100 täglichen Testumgebungen ohne manuellen Eingriff ermöglicht
\item \justifying Erstellung von \textbf{Grafana}-Dashboards mit \textbf{Flux} zur Überwachung der Leistung und des Zustands von über 20 wichtigen Backend-Diensten
\end{itemize}

\vspace{0.75em}

\divider

\vspace{0.75em}

\cvevent{Senior Software Engineer - Backend}{Innovative Solutions \href{https://shohagh.com/}{\faGlobe}}{August 2018 -- August 2019}{Dhaka, Bangladesch}
\begin{itemize}
\item \justifying Leitung der Entwicklung des \textbf{Freeway}-Projekts: Online-\textbf{Ticket-Reservierungs}- und -\textbf{Kaufsystem} für Busse mit mehr als 30 verschiedenen Zielen von Shohagh Paribahan Limited, einem führenden Busdienst in Bangladesch
\item \justifying Verwalten der Entwicklung von \textbf{Spring Boot} + \textbf{MySQL}-basierten Backend-Diensten und APIs, die den gesamten Lebenszyklus abdecken: Anforderungsanalyse, Design, Implementierung, Testen, Refactoring und Bereitstellung in Entwicklungs- und Produktionsumgebungen mit \textbf{Docker} und \textbf{TeamCity}
\end{itemize}

\vspace{0.75em}

\divider

\vspace{0.75em}

\cvevent{Software Engineer}{BizMotion Limited \href{https://www.biz-motion.com/}{\faGlobe}}{Februar 2013 -- Mai 2018}{Dhaka, Bangladesch}
\begin{itemize}
\item \justifying Konzeption und Leitung der Entwicklung des \textbf{SMC-Program} projekts, mit dem über 500 Pharmareferenten Umfragen, Bestandsaktualisierungen und Schulungsmaterialien über ein \textbf{Spring Boot} + \textbf{PostgreSQL} Backend verwalten können
\item \justifying Entwicklung von Backend-Diensten und APIs für das Projekt \textbf{BizPharma}, eine Lösung zur Automatisierung des Außendienstes, die von mehr als 20 \textbf{FMCG} Unternehmen zur Rationalisierung der Arbeitsabläufe in den Bereichen Lagerhaltung, Verkauf und Vertrieb genutzt wird
\item \justifying Full-Stack-Entwickler des VoIP Dialer CMS-Projekts - \textbf{Spring-Framework} und \textbf{PostgreSQL}-basierte Webanwendung für rollenbasiertes Zugangs- und Betriebsmanagement von VoIP-Ressourcen
\end{itemize}

\vspace{2em}

\cvsection{Ausbildung}

\cvevent{M.Sc.\ in Data Science}{Berliner Hochschule für Technik}{Oktober 2019 -- März 2025}{Berlin, Deutschland}

\divider

\cvevent{B.Sc.\ in Computer Science and Engineering}{Shahjalal University of Science and Technology}{Januar 2008 -- Dezember 2011}{Sylhet, Bangladesch}

\vspace{2em}

\cvsection{Veröffentlichungen}
\begin{justifying}
\nocite{saha2025pseudonymization}
\printbibliography[heading=none]
\end{justifying}

\end{paracol}

\end{document}